% This is "sig-alternate.tex" V2.0 May 2012
% This file should be compiled with V2.5 of "sig-alternate.cls" May 2012
%
% This example file demonstrates the use of the 'sig-alternate.cls'
% V2.5 LaTeX2e document class file. It is for those submitting
% articles to ACM Conference Proceedings WHO DO NOT WISH TO
% STRICTLY ADHERE TO THE SIGS (PUBS-BOARD-ENDORSED) STYLE.
% The 'sig-alternate.cls' file will produce a similar-looking,
% albeit, 'tighter' paper resulting in, invariably, fewer pages.
%
% ----------------------------------------------------------------------------------------------------------------
% This .tex file (and associated .cls V2.5) produces:
%       1) The Permission Statement
%       2) The Conference (location) Info information
%       3) The Copyright Line with ACM data
%       4) NO page numbers
%
% as against the acm_proc_article-sp.cls file which
% DOES NOT produce 1) thru' 3) above.
%
% Using 'sig-alternate.cls' you have control, however, from within
% the source .tex file, over both the CopyrightYear
% (defaulted to 200X) and the ACM Copyright Data
% (defaulted to X-XXXXX-XX-X/XX/XX).
% e.g.
% \CopyrightYear{2007} will cause 2007 to appear in the copyright line.
% \crdata{0-12345-67-8/90/12} will cause 0-12345-67-8/90/12 to appear in the copyright line.
%
% ---------------------------------------------------------------------------------------------------------------
% This .tex source is an example which *does* use
% the .bib file (from which the .bbl file % is produced).
% REMEMBER HOWEVER: After having produced the .bbl file,
% and prior to final submission, you *NEED* to 'insert'
% your .bbl file into your source .tex file so as to provide
% ONE 'self-contained' source file.
%
% ================= IF YOU HAVE QUESTIONS =======================
% Questions regarding the SIGS styles, SIGS policies and
% procedures, Conferences etc. should be sent to
% Adrienne Griscti (griscti@acm.org)
%
% Technical questions _only_ to
% Gerald Murray (murray@hq.acm.org)
% ===============================================================
%
% For tracking purposes - this is V2.0 - May 2012

\documentclass{article}

\usepackage{graphicx}
\usepackage[space]{grffile}
\usepackage{latexsym}
\usepackage{amsfonts,amsmath,amssymb}
\usepackage{url}
\usepackage[utf8]{inputenc}
\usepackage{hyperref}
\hypersetup{colorlinks=false,pdfborder={0 0 0}}
\usepackage{textcomp}
\usepackage{longtable}
\usepackage{multirow,booktabs}
\newcommand{\truncateit}[1]{\truncate{0.8\textwidth}{#1}}
\newcommand{\scititle}[1]{\title[\truncateit{#1}]{#1}}
\newcommand\mathplus{+}

\usepackage{multicol}
\usepackage[normalem]{ulem} % for strikeout
\usepackage{xcolor,colortbl} % for cell color
\usepackage{flushend} % for ref columns equalization
\usepackage{csquotes} % for quotes

%% duplicates
\usepackage{graphicx}
\usepackage{booktabs}  % for tab rules
\usepackage{multirow}

\pagenumbering{gobble} % remove page no.
\CopyrightYear{2016}



%% Here go all packages compatible with Authorea
\usepackage[normalem]{ulem} % for strikeout
\usepackage{xcolor,colortbl} % for cell color
\usepackage{flushend} % for ref columns equalization
\usepackage{csquotes} % for quotes
\usepackage{authblk} % for authors multi affiliations

\pagenumbering{gobble} % remove page no.

\title{Migrating to Stack Overflow: A Dataset of Q\&A from Legacy Developer Forums}

\usepackage{authblk}

\author[1]{Fabio Calefato\thanks{fabio.calefato@uniba.it}}
\author[1]{Filippo Lanubile\thanks{filippo.lanubile@uniba.it}}
\author[2]{Nicole Novielli\thanks{nicole.novielli@uniba.it}}
\affil[1]{Department of Computer Science, University of Bari, Italy}
\affil[2]{Department of Computer Science, University of Bari, Italy}

\renewcommand\Authands{ and }
}


\begin{document}

\maketitle

\begin{abstract}
Rielaborare quello da ICSE 2015
As more and more developer support forums abandon their legacy websites to move onto Stack Overflow, lots of user-generated knowledge is at risk of being left behind.

Recently, more and more developer communities are abandoning their legacy support forums, moving onto Stack Overflow. The motivations are diverse, yet they typically include achieving faster response time and larger visibility through the access to a modern and very successful infrastructure.

we add to the body of evidence of existing research on best answer prediction and show that, from a technical perspective, the content from existing developer forums might be automatically migrated to the Stack Overflow, although most of forums do not allow to mark a question as resolved, a distinctive feature of modern question answering sites. For this purpose, we trained a binary classifier with data from Stack Overflow and then tested it with data scraped from Docusign, a developer forum that has recently completed the move. 

\end{abstract}

\bibliographystyle{abbrv}

\input{section_Introduction__}

\input{section_Background__}

\input{section_Experiment__}

\input{section_Discussion__}

\input{section_Limitations__}

\input{section_Conclusions__}

\section*{Acknowledgements}
This work is partially funded by the project \enquote*{Investigating the Role of Emotions in Online Question \& Answer Sites}, funded by MIUR (Ministero dell{'}Universit{\`{a}} e della Ricerca) under the program \enquote{Scientific Independence of young Researchers} (SIR).

\bibliography{./bibliography/biblio}

\end{document}
